\documentclass[preprint,aps,onecolumn,preprintnumbers,amsmath,amssymb,nofootinbib,superscriptaddress]{revtex4-1}

\vspace{5mm}

%=========================================================================
\usepackage{slashed}
\usepackage{graphicx}
\usepackage{amssymb}
\usepackage{mathtools}
\usepackage{bbold}
\usepackage{amssymb,latexsym}
\usepackage{amsmath,amsbsy,bbm}
\usepackage{multirow}
\usepackage[vcentermath]{youngtab}
\usepackage{nicefrac}
\usepackage[perpage]{footmisc}
\usepackage{wrapfig,lipsum,booktabs}
\usepackage{caption}
\usepackage{subcaption}
\usepackage{graphicx}
\usepackage{cjhebrew}
\usepackage{cleveref}
\usepackage[utf8]{inputenc}
\usepackage{soul}

\usepackage{booktabs}

%\usepackage{nicematrix}
\usepackage{floatrow}
%\usepackage[dvipsnames]{xcolor} 
\usepackage{tikz}
\usetikzlibrary{fit}

\tikzset{highlight/.style={rectangle,
fill=black!25,
rounded corners = 0.5 mm,
inner xsep=3.1pt,
fit=#1}}

\tikzset{highlighted/.style={rectangle,
fill=black!5,
rounded corners = 0.5 mm,
inner xsep=3.1pt,
fit=#1}}

%=========================================================================
\newfloatcommand{capbtabbox}{table}[][0.45\textwidth]
%=========================================================================
\newcommand{\es}{1\text{\scriptsize s}}
\newcommand{\zs}{2\text{\scriptsize s}}
\newcommand*{\mprime}{^{\prime}\mkern-1.2mu}
\newcommand{\largescale}{\ensuremath{\Lambda_\text{Hi}}}
\newcommand{\lc}{\ensuremath{\Lambda_c}}
\newcommand{\fm}{\ensuremath{\,\text{fm}^{-1}}}
\newcommand{\abb}{\mbox{\ensuremath{A\oplus 1}}}
\newcommand{\red}[1]{\textcolor{red}{#1}} 
\newcommand{\green}[1]{\textcolor{green}{#1}} 
\newcommand{\blue}[1]{\textcolor{blue}{#1}} 
\newcommand{\lec}{C^\Lambda}
\newcommand{\led}{D^\Lambda}
\newcommand{\ddrei}[1]{\delta_{\tiny \Lambda}^{(3)}\!\big(#1\big)}
\newcommand{\wrt}{\textit{w.r.t.}}
\newcommand{\etc}{\textit{etc.}}
\newcommand{\eg}{\textit{e.g.}}
\newcommand{\ie}{\textit{i.e.}}
\newcommand{\cf}{\textit{cf.}}
\newcommand{\viz}{\textit{viz.}}
\newcommand{\eftnopi}{\mbox{EFT$(\not \! \pi)$}}
\newcommand{\ve}[1]{\ensuremath{\boldsymbol{#1}}}
\newcommand{\rms}[1]{\ensuremath{\langle r(#1)\rangle}}
\newcommand{\ls}{\ve{L}\cdot\ve{S}}
\newcommand{\be}{\begin{equation}}
\newcommand{\ee}{\end{equation}}
\newcommand{\bra}{\big\langle}
\newcommand{\ket}{\big\rangle}
\newcommand{\vcl}[1]{\ensuremath{\bar{\boldsymbol{r}}_\text{\tiny #1}}}
\newcommand{\vsp}[1]{\ensuremath{\boldsymbol{r}}_\text{\tiny #1}}
\newcommand{\la}{\label}
\newcommand{\Pe}{\text{\cjRL{p|}}}
\newcommand{\figref}[1]{fig.~(\ref{#1})}
\newcommand{\tabref}[1]{table~(\ref{#1})}
\newcommand{\ccite}[1]{\cite{#1}}
\newcommand{\eref}[1]{\cref{#1}}
\newcommand{\pd}[1]{\frac{\partial}{\partial#1}}
\newcommand{\abcd}{\ensuremath{a_\text{\tiny (abcd)}}}
\newcommand{\abcc}{\ensuremath{a_\text{\tiny (acbc)}}}
\newcommand{\add}{\ensuremath{a_\text{\tiny (abab)}}}
\newcommand{\bd}{\ensuremath{B_2}}
\newcommand{\bt}{\ensuremath{B_3}}



%\definecolor{blue}{HTML}{4169E1}
%\definecolor{red}{HTML}{DC143C}
%\definecolor{green}{HTML}{2E8B57}
%\definecolor{mandarin}{HTML}{FF9933}

%\newcommand{\mpv}[1]{\textcolor{purple}{#1}}

%\newcommand{\corr}[1]{\textcolor{blue}{#1}}

%\newcommand{\LC}[1]{\textcolor{mandarin}{#1}}
%\newcommand{\JK}[1]{\textcolor{green}{#1}}
%\newcommand{\note}[1]{\textbf{\textcolor{gray}{#1}}}

%\newcommand{\remove}[1] {\textcolor{red}{\sout{#1}}}
%\newcommand{\comment}[2] {\textcolor{green}{[\textbf{comment - #1}: {#2}]}}
%\newcommand{\remark}[2] {\textcolor{orange}{[\textbf{\textsf{remark - #1}}: {\textsf{#2}}]}}
%\newcommand{\highlight}[1] {\textcolor{red}{{#1}}}
%\newcommand{\repl}[2] {\textcolor{red}{\st{#1}}\textcolor{blue}{#2}}
%\newcommand{\place}[1] {\textcolor{blue}{{#1}}}
%\newcommand{\cmm}[1] {\textcolor{green}{{#1}}}

%=========================================================================
\usepackage[normalem]{ulem}



\begin{document}

\title{On the effect of particle identity on inter-cluster interactions:\\
dimer-dimer}

\author{Rakshanda Goswami}
\author{Udit Raha}
\address{Department of Physics, Indian Institute of Technology Guwahati, Guwahati 781039, India}
\author{Johannes Kirscher}
\address{Department of Physics, SRM University - AP, Amaravati 522502, Andhra Pradesh, India}
\date{\today}

\date{\today}

\begin{abstract}
The effects of particle statistics, proximity to the two-particle unitary limit,
and the short-distance structure of the particle-particle interaction on
elastic scattering between two composite dimers is analyzed. Specifically, we
obtain relative $S$-wave scattering lengths for four distinguishable particles of $\abcd=xx()()$.
If three of four fermions are distinguishable and the fourth is identical to one of them,
$\abcc=xx()()$. For two-component fermions, we recover the well-known universal ratio between
dimer-dimer and particle-particle scattering length
$s\add\approx0.6\,a$.
The dependence of all dimer-dimer scattering lengths on short-distance interactions (two- and three-body contacts)
for which the employed one-channel resonating-group technique is appropriate is finite and unaffected by a
change in the number of four-body bound states as induced by the renormalization-group transformation.
We provide an intuitive interpretation of the increasing attraction/repulsion between the dimers as an
effect of particle statistics by visualizing the ensuing effective local and non-local energy-dependent
dimer-dimer potentials in coordinate space.
\end{abstract}

\maketitle
%=============================================================================

\section{Introduction}
This is the first in a series of four articles in which we employ the resonating-group 
technique -- also referred to as folding-model\footnote{Refs.~\cite{SINHA19751,Satchler:1979ni}~review the
original ideas and the relation to the optical model. Refs.~\cite{Naidon_2016,Kanada-Enyo:2020zzf,Rokash:2016tqh}
~are recent refinements/applications of the methodology to other than nuclear systems.} -- in order
to parametrize effective interactions between composites which are bound in a minimal
theory for two- and three-body systems with the parameters of the latter.
In this first part of the series, we introduce the technique as it yields an interaction
between the centers of mass of two composites (henceforth referred to as dimers), each of which
forming a state of two assumed-fundamental particles with identical binding energy $\bd$.
We will justify the seemingly drastic assumption of dimers being unaffected in the course of
a low-energy scattering event described by this effective potential through a comparison with
numerical solutions of the full four-body problem~\cite{Petrov_2004}
\footnote{The problem was analyzed in Refs.~\cite{Elhatisari_2017,Schafer:2022hzo,Deltuva_2022}~(each of which represents a
particular numerical approach) in a regime of }.
In the second and third parts we will apply
the method to the trimer-trimer and tetramer-tetramer systems, respectively.
In the fourth article, we treat the number of particles as a parameter and present the dependence
of an effective ($A$-body)-($B$-body) potential on the two- and three-body coupling strengths, and we
extend the framework to include three-fragment channels.

With this work, we attempt to contribute to all earlier and ongoing efforts to analyze the features
of few- and many-body quantum systems in terms of a small set of fundamental parameters.
A set of (effective) field theories (EFTs) -- crudely understood as the description of a system by an appropriate set
of degrees of freedom (DoF), symmetries, interaction structure, and with a defined, so-called breakdown
scale beyond which typically other DoFs are resolved -- replacing a single theory from which everything can
be derived has been successful especially in identifying universal behavior.
In principle, the EFT framework
prescribes how to express the parameters of one theory in terms of its underlying one,
e.g. the nuclear coupling strengths in terms of Standard-model parameters, or the strength of the attraction
between $\alpha$ clusters in terms of nuclear masses and couplings. In practice, the change from one set of Dofs to another
precludes a rigorous derivation of such relations.

Here, we advance the folding/resonating-group potentials to provide such relations when dimer, trimer, tetramer, \&c DoFs
are composites of a certain class of point fermions/bosons whose pair interaction is vanishingly small compared with the
two-body scale it produces. Nucleons, helium-isotope atoms, and a set of trapped cold atoms fall into this category. For nucleons,
in particular, the application of such zero-range interactions to few-body systems in combination with renormalization-group techniques
allowed to discriminate between universal and characteristic properties of larger-in-number systems; prominently, the 
sheer existence of an excited, resonant state and a deeper ground state of the $J^\pi=0^+$ $\alpha$ nucleus are universal
consequences of a large nucleon-nucleon scattering length compared with the nuclear force range ($\approx$ inverse $\pi$-meson mass).
The magnitude of the binding energy and the location of the resonance relative to the 3-helium-neutron and 3-hydrogen-proton thresholds
is a characteristic of the nuclear substructure with is encoded in the binding energy of the nuclear trimers (with~\bt, we denote the
ground-state energy of the generic trimer and not a specific datum).
To classify observables involving more than four nucleons -- in general, systems comprised of more particles than accessible internal
degrees of freedom -- analogously as universal or characteristic is desirable in order to, first, understand features like mass gaps
and shell structure from a minimal set of underlying parameters, and subsequently, to employ this insight for a reliable, model-independent
prediction of unknown and hard-to-measure quantities like, e.g., few-neutron states.

While keeping these features of nuclei in mind, we analyze the essence of the problem: the effect of particle statistics on the effective
composite-composite interaction with the bound composites being universally correlated to an approximately scale-free two-body system and
, for systems with more than three distinguishable particles, a finite three-body scale. A pioneering study~\cite{Petrov_2004}~found a weakening
of the attraction between two dimers whose amount induces a ratio of dimer-dimer and atom-atom scattering lengths of $0.6$ in the absence of
a low-energy three-atom scale. If the internal space accessible to the fermions equals three or four, bosonic behavior enters in form of a finite
scale that is characteristic of the system under consideration. The de- or independence of cluster-cluster scattering on this scale and how its
existence weakens or strengthens the dimer-dimer interaction compared with the two-component-fermion system is the main result of this article
besides the introduction of the folding technique as part of the EFT framework to obtain model-independent inter-cluster interactions.
From the latter, we derive analytically a potential which depends solely on the separation of the centers of mass of the two dimers, their binding
energy~\bd, and a regularization parameter (henceforth denoted as $\lambda$) that is introduced when the EFT is renormalized using~\bd~(and~\bt~when
appropriate) as {\it the} constraint(s). The need for 

\section{contact EFT $\to$ inter-cluster EFT}

\section{leading-order description of the dimer-dimer interaction}

\subsection{(abab)}
\subsection{(abcd)}
\subsection{(abcc)}

\section{summary}

\section{Appendix: folding contact potentials}

\bibliographystyle{apsrev}
\bibliography{dd_scatt_universality.bib}
\end{document}